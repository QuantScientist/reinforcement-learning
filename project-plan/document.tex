\documentclass[a4paper,12pt]{article}

\usepackage[T1]{fontenc}        % cleaner font
\usepackage[utf8]{inputenc}     % UTF-8 support
\usepackage{amsmath}            % more math symbols
\usepackage{amssymb}            % allow math symbols of form \mathbb{...}
\usepackage{graphicx}           % enhanced graphics support
\usepackage{epstopdf}           % automagically turn eps to pdf, for gnuplot
\usepackage{subcaption}         % subfigure support
\usepackage[dvipsnames]{xcolor} % syntax coloring support
\usepackage{listings}           % programming language support
\usepackage{algorithmic}        % pseudocode support
\usepackage{algorithm}
\usepackage{cancel}             % cancel out terms in division
\usepackage{parskip}            % enable spacing between paragraphs
\usepackage{cases}              % enable math function definitions with cases
\usepackage{titling}            % allow adjustment of document title
\usepackage{fullpage}           % 1 inch margins
\usepackage{bm}                 % bold math fonts for vectors and stuff
\usepackage{nicefrac}           % Nicer looking fractions
\usepackage{floatflt}           % Allows text to float around figures / tables
\usepackage{hyperref}           % Clickable references, citations etc
\usepackage[
abbreviate=false,
backend=biber,
sortcites=true,
sorting=nyt,
sortlocale=en_US,
style=numeric-verb,
maxnames=10
]{biblatex}                     % Bibliography support

%\usepackage[finnish]{babel}    % use Finnish for hyphenation
%\usepackage{color}             % colored text?

% Adjust title vertical position
\setlength{\droptitle}{-2cm}

% Set up syntax highlighting for programming languages
\lstloadlanguages{Ruby}
\lstset{%
basicstyle=\ttfamily\bfseries\footnotesize,
commentstyle = \ttfamily\color{orange},
keywordstyle=\ttfamily\color{blue},
stringstyle=\color{red},
showstringspaces=false,
frame=trbl,
}

% Custom probability macros
\def\ci{\perp\!\!\!\perp}              % Independence symbol
\newcommand{\jpr}[2]{P(#1 \, , \, #2)} % Joint probability
\newcommand{\cpr}[2]{P(#1 \, | \, #2)} % Conditional probability

% Custom topological macros
\newcommand\opn{\mathrel{\ooalign{$\subset$\cr        % open set \opn
  \hidewidth\hbox{$\circ\mkern.5mu$}\cr}}}
\newcommand\cls{\mathrel{\ooalign{$\subset$\cr        % closed set \cls
  \hidewidth\raise.225ex\hbox{$\text{{\scriptsize c}}\mkern2mu$}\cr}}}

% Custom handy macros for tuples and sets in maths
\newcommand{\setof}[1]{\ensuremath{\left \{ #1 \right \}}}
\newcommand{\tuple}[1]{\ensuremath{\left \langle #1 \right \rangle }}

% Finnish-style limits after integration symbol (by Martti Nikunen)
\newcommand{\vii}{\mathop{\Big/}}
\newcommand{\viiva}[2]{\vii\limits_{\!\!\!\!{#1}}^{\>\,{#2}}}

% Configure bibliography database
\bibliography{sources.bib}

\begin{document}

\title{Display advertisement}
\author{Eric Andrews}

\maketitle

\section*{Preliminary abstract}
Many content producers on the Internet fund their operations via display
advertising. In this business model, ads are placed around or within the
content with the hopes that visiting users will click on the ads, thereby
generating revenue to the content producer.

Unfortunately just showing the same ads to all users tends to be ineffective,
since the interests of users are often very diverse. Moreover, placing too many
ads into the same content may annoy users, and the effectiveness of those ads
may decline.

In targeted advertising, we attempt to show users the ads they are most likely
to be interested in, by maintaining a profile of the users. These profile are
built, for example, by keeping track of the interactions the user has had with
the site.

My plan is to first study the problem of targeted advertising from a contextual
$n$-armed bandits viewpoint, focusing on the exploration vs. exploitation
dilemma. The paper on Thompson Sampling \cite{chapelle2011empirical} seems like
a good place to start. In the Mortal Multi-Armed Bandits paper
\cite{chakrabarti2008mortal} they consider the case where bandits (ads) have
limited lifespans and new ones emerge. This also seems like a pretty relevant
aspect to take into consideration.

I will then study the problem from the full reinforcement learning viewpoint.
The paper titled "Concurrent Reinforcement Learning from Customer Interactions"
\cite{silver2013concurrent} seems pretty relevant in this case. The RL problem
presented in the paper is a very specific kind: many users interact with the
system concurrently, and we model RL 'state' as history of a user's
interactions. The approach considered uses a modification to temporal
difference learning, so I will have to delve into that as well.


\nocite{*}

\printbibliography

\end{document}

% ---- EXAMPLES BELOW ----

% -- Example of figure for imagename.eps
% \begin{figure}
%     \includegraphics[scale=0.5]{imagename}
%     \caption*{Caption be here.}
% \end{figure}

% -- Example of floating figure for imagename.eps
% \begin{floatingfigure}[r]{0.49\textwidth}
%   \includegraphics[scale=0.31]{imagename}
%   \caption*{Caption be here.}
% \end{floatingfigure}

% -- Example of two figures side-by-side
% \begin{figure}[h]
%   \centering
%   \begin{subfigure}{.5\textwidth}
%     \centering
%     \includegraphics[width=.9\linewidth]{fig1.eps}
%     \caption*{Caption here}
%   \end{subfigure}%
%   \begin{subfigure}{.5\textwidth}
%     \centering
%     \includegraphics[width=.9\linewidth]{fig2.eps}
%     \caption*{Caption be here}
%   \end{subfigure}
%   \caption{Shared caption}
%   \label{fig:fighere}
% \end{figure}

% -- Remove pagination
% \thispagestyle{empty}
% \pagestyle{empty}

% -- Example of code listing
%\lstinputlisting[language=Ruby, caption={file.rb}]{./file.rb}

% -- Example of cases-environment
% $$
% f(x) =
% \begin{cases}
%     x^2 & \text{if } x > 0 \\
%     0 & \text{otherwise } \\
% \end{cases}
% $$

% -- Multiletter variables in math mode
%$$
%  \textit{WM}
%$$

% -- Example of multicolumn row in tabular environment
% \multicolumn{3}{|c|}{Cell spanning three columns}
